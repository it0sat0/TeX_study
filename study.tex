\documentclass[a4j]{jarticle}
\usepackage{listings}
\usepackage[dvipdfmx]{graphicx}
\usepackage{here}
\usepackage{ascmac}

\begin{document}
\begin{titlepage}
\title{\LaTeX の使い方~基本編~}
\author{
\\
情報理工学部2回生 伊藤聡子 
\\
}
\date{2018年2月5日}
\maketitle
\thispagestyle{empty}
\newpage
\end{titlepage}

\tableofcontents
\newpage

\section{概要}
\LaTeX とはなんぞや? そう思う人はかなりいると思います。\\
\\
  \LaTeX (ラテック、ラテフ)とは、レスリー・ランポートによって開発されたテキストベースの組版処理システムである。電子組版ソフトウェア TeX にマクロパッケージを組み込むことによって構築されており、単体の \TeX に比べて、より手軽に組版を行うことができるようになっている。 と表記できない場合は“\LaTeX ”と表記する。\\
 なお、\LaTeX を基にアスキーが日本語処理に対応させたものとして日本語 \LaTeX が、さらに縦組み処理にも対応させたものとして pLaTeX がある。\\
 専門分野にもよるが、学術機関においては標準的な論文執筆ツールとして扱われている。
\begin{flushright}
 by.wikipedia
\end{flushright}
 簡単に言うと、上手い感じに文章や図を配置してPDFなどに出力できるソフトのことです。\\
 この冊子では、\LaTeX の中でも基本的な、書くのに必要最小限のことだけを紹介していきます。

\section{文章構造について}

\subsection{表紙のつくり方}
\LaTeX は文章の中でも主に、レポートや論文を書くときに使われます。そんな時、いきなり本文から始めるなんてことはありえません。初めにタイトルや著者名、日付が書かれた表紙がおかれることがほとんどです。まずは、そんな表紙のつくり方から始めたいと思います。\\

  \begin{table}[H]
    \begin{center}
      \caption{表紙を作るときに使うコマンド}
      \begin{tabular}{|c||c|} \hline
        用途 & コマンド  \\ \hline \hline
	   タイトル & \verb#\title# \\ \hline
        著者名 & \verb#\author# \\ \hline
        日付 & \verb#\date# \\ \hline
        タイトルを出力 & \verb#\maketitle# \\ \hline
      \end{tabular}
    \end{center}
  \end{table}

 表紙を作るとき、主に以上のコマンドを使用します。\\
 実際に書いてみると、以下のようになります。\\

\begin{screen}
\begin{verbatim}
\documentclass[11pt,a4j]{jarticle}

\begin{document}
\title{タイトル}
\author{著者名}
\date{日付}

\maketitle

以下、文章を書く

\end{document}
\end{verbatim}
\end{screen}


\subsection{項目のつくり方}

\subsection{箇条書きのつくり方}

\subsection{ページ番号のつくり方}


\section{図について}


\section{特殊文字・記号について}

  \begin{table}[H]
    \begin{center}
      \caption{特殊文字・記号を表示させるコマンド}
      \begin{tabular}{|c||c|} \hline
        用途 & コマンド  \\ \hline \hline
	   タイトル & \verb#\title# \\ \hline
        著者名 & \verb#\author# \\ \hline
        日付 & \verb#\date# \\ \hline
        タイトルを出力 & \verb#\maketitle# \\ \hline
      \end{tabular}
    \end{center}
  \end{table}


\section{おまけ}

\subsection{目次を作る}
\verb#\tableofcontents#

\subsection{コメントアウトする}

\subsection{指定した範囲を枠で囲む}
  \verb#\usepackage{ascmac}#

\subsection{プログラムを表示させる}
  \verb#\usepackage{listings}#

\end{document}