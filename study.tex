\documentclass[a4j]{jarticle}
\usepackage{listings}
\usepackage[dvipdfmx]{graphicx}
\usepackage{here}
\begin{document}
\begin{titlepage}
\title{\LaTeX の使い方~基本編~}
\author{
\\
情報理工学部2回生 伊藤聡子 \\
}
\date{2018年2月5日}
\maketitle
\thispagestyle{empty}
\newpage
\end{titlepage}

\tableofcontents
\newpage

\section{概要}
\LaTeX とはなんぞや? そう思う人はかなりいると思います。\\
\\
 \LaTeX (ラテック、ラテフ)とは、レスリー・ランポートによって開発されたテキストベースの組版処理システムである。電子組版ソフトウェア TeX にマクロパッケージを組み込むことによって構築されており、単体の \TeX に比べて、より手軽に組版を行うことができるようになっている。 と表記できない場合は“\LaTeX ”と表記する。
 なお、\LaTeX を基にアスキーが日本語処理に対応させたものとして日本語 \LaTeX が、さらに縦組み処理にも対応させたものとして pLaTeX がある。
 専門分野にもよるが、学術機関においては標準的な論文執筆ツールとして扱われている。
\begin{flushright}
 by.wikipedia
\end{flushright}
 簡単に言うと、上手い感じに文章や図を配置してPDFなどに出力できるソフトのことです。

\section{文章構造について}
\subsection{表紙のつくり方}
\subsection{項目のつくり方}
\subsection{箇条書きのつくり方}
\subsection{ページ番号のつくり方}
\section{フォントについて}
\section{数式について}
\section{図について}
\section{特殊文字・記号について}
\end{document}