\documentclass[a4j]{jarticle}
\usepackage{listings}
\usepackage[dvipdfmx]{graphicx}
\usepackage{here}
\usepackage{ascmac}

\begin{document}
%
%表紙の作成
%
\begin{titlepage}
\title{\LaTeX の使い方 \\
~基本編~}
\author{
\\
情報理工学部2回生 伊藤聡子 
\\
}
\date{2018年2月5日}
\maketitle
\thispagestyle{empty}
\newpage
\end{titlepage}

%
%目次の作成
%
\tableofcontents
\newpage

%
%概要
%
\section{概要}
\LaTeX とはなんぞや? そう思う人はかなりいると思います。\\
\\
 \LaTeX (ラテック、ラテフ)とは、レスリー・ランポートによって開発されたテキストベースの組版処理システムである。電子組版ソフトウェア TeX にマクロパッケージを組み込むことによって構築されており、単体の \TeX に比べて、より手軽に組版を行うことができるようになっている。\\s
 なお、\LaTeX を基にアスキーが日本語処理に対応させたものとして日本語 \LaTeX が、さらに縦組み処理にも対応させたものとして pLaTeX がある。\\
 専門分野にもよるが、学術機関においては標準的な論文執筆ツールとして扱われている。
\begin{flushright}
 by.wikipedia
\end{flushright}
 簡単に言うと、上手い感じに文章や図を配置してPDFなどに出力できるソフトのことです。\\
 この冊子では、\LaTeX の中でも基本的な、書くのに必要最小限のことだけを紹介していきます。\\
 最後にはおまけとして、レポートを書く上で知って置いたら便利だというものを載せているので、興味がある人は見てみてください。
\newpage

%
%環境構築
%
\section{環境構築}
\subsection{Windows編}
\subsection{Mac編}
青のり君にまかせた
\subsection{Linux編}
ここも青のり君にまかせた
\newpage

%
%文章構成
%
\section{文章構造について}
%
%表紙
%
\subsection{表紙のつくり方}
\LaTeX は文章の中でも主に、レポートや論文を書くときに使われます。そんな時、いきなり本文から始めるなんてことはありえません。初めにタイトルや著者名、日付が書かれた表紙がおかれることがほとんどです。まずは、そんな表紙のつくり方から始めたいと思います。\\

  \begin{table}[H]
    \begin{center}
      \caption{表紙を作るときに使うコマンド}
      \begin{tabular}{|c|c|} \hline
        用途 & コマンド  \\ \hline \hline
	   タイトル & \verb#\title# \\ \hline
        著者名 & \verb#\author# \\ \hline
        日付 & \verb#\date# \\ \hline
        タイトルを出力 & \verb#\maketitle# \\ \hline
      \end{tabular}
    \end{center}
  \end{table}

 表紙を作るとき、主に以上のコマンドを使用します。\\
 実際に書いてみると、以下のようになります。\\

\begin{screen}
\begin{verbatim}
\documentclass[11pt,a4j]{jarticle}

\begin{document}
\title{タイトル}
\author{著者名}
\date{日付}

\maketitle

以下、文章を書く

\end{document}
\end{verbatim}
\end{screen}

 これを書くことで、この冊子の表紙のような表紙を作ることが出来ます。\\
 \verb#表紙を作るときは、\maketitkeは必ず必要になります。これがなければ、せっかく書いても表紙はできませんので注意してください。#\\
 \verb#また、タイトルが長かったり、著者が複数人いる場合に改行したいときは、改行したい部分に\\(バックスラッシュ2つ)を入れると行が変わります。#
%
%項目
%
\subsection{項目のつくり方}
本を読んでいると、第一章、第二章と、ストーリーがシーンや大きな出来事ごとに章が変わっていたり、教科書を見ていると、第一章一節、二節と、一つの単元の中でも細かく分かれているということがあると思います。\\
 ここでは、章や節のつくり方を説明します。これを使うことで、レポートや報告書を書くときにとても見やすくなります。どのような見た目になるかというと、この冊子の項目のようになります。「1.概要」や、「2.2項目のつくり方」のような感じです。\\
 では、

%
%箇条書きのつくり方
%
\subsection{箇条書きのつくり方}
\LaTeX での箇条書きには,3つの種類があります.これから先にレポートや報告書を書く上で使い分けが必要になってくると思うので,全て説明します.\\
まず,箇条書きの基本的な書き方は,以下のようになります.
\begin{screen}
\begin{verbatim}
\begin{○○}
\item 一つ目の項目
\item 二つ目の項目
\end{○○}
\end{verbatim}
\end{screen}
「○○」の中には,黒丸や見出し付きなどのどのような箇条書きにしたいかによって違うものが入ります.それでは,それぞれの説明に入ります.

\subsubsection{黒丸}
黒丸の場合は以下のような表示になります.
\begin{itemize}
\item 一つ目
\item 二つ目
\end{itemize}
実際に書いてみると,以下のようになります.
\begin{screen}
\begin{verbatim}
\begin{itemize}
\item 一つ目
\item 二つ目
\end{itemize}
\end{verbatim}
\end{screen}

\subsubsection{数字}
数字の場合は以下のような表示になります.
\begin{enumerate}
\item 一つ目
\item 二つ目
\end{enumerate}
実際に書いてみると,以下のようになります.
\begin{screen}
\begin{verbatim}
\begin{enumerate}
\item 一つ目
\item 二つ目
\end{enumerate}
\end{verbatim}
\end{screen}

\subsubsection{見出しつき}
見出しつきのは以下のような表示になります.
\begin{description}
\item [価格] 一つ目
\item [値段] 二つ目
\end{description}
実際に書いてみると,以下のようになります.
\begin{screen}
\begin{verbatim}
\begin{description}
\item [価格] 一つ目
\item [値段] 二つ目
\end{description}
\end{verbatim}
\end{screen}

%
%ページ番号の付け方
%
\subsection{ページ番号の付け方}

\newpage

%
%図について
%
\section{図について}

\newpage

%
%特殊文字・記号について
%
\section{特殊文字・記号について}

  \begin{table}[H]
    \begin{center}
      \caption{特殊文字・記号を表示させるコマンド}
      \begin{tabular}{|c||c|} \hline
        用途 & コマンド  \\ \hline \hline
	   タイトル & \verb#\title# \\ \hline
        著者名 & \verb#\author# \\ \hline
        日付 & \verb#\date# \\ \hline
        タイトルを出力 & \verb#\maketitle# \\ \hline
      \end{tabular}
    \end{center}
  \end{table}

\newpage
%
%おまけ
%
\section{おまけ}

%
%目次
%
\subsection{目次を作る}
\verb#\tableofcontents#を使います.

%
%コメントアウト
%
\subsection{コメントアウトする}
書いている文章が長くなってくると、読み返したときにどこに何を書いたのかわからなくなってくることがあるかもしれません。そんなときに便利なのが、コメントアウト。コメントアウトとは、プログラムの途中に、プログラムに影響のないようにメモ書きすることを言います。文章を書いているだけの\LaTeX でも、このコメントアウトは存在します。\\
 デフォルトではコメントアウトした部分は赤色になるので、ぱっと見でもとても分かりやすいと思います。 プログラムを表示させるためには、コメントアウトしたいのはじめに「\% (パーセント)」を付けるだけです。この資料を作っているときも、これを乱用しています。\\
 具体的な使い方は、

\begin{screen}
\begin{verbatim}

%石山にぼ
%ゴッドこなつ神ゴッド

\end{verbatim}
\end{screen}

 このようになります。文中でも「\% 」を入れると、その行はそのあと全部がコメントアウトされます。\\
 また、コメントアウトはエラーが発生した時にエラーの原因を探すためにも使えます。

\subsection{指定した範囲を枠で囲む}
  \verb#\usepackage{ascmac}#を使います。これがなければエラーを吐かれてしまうので、気を付けましょう。\\


\subsection{プログラムを表示させる}
  \verb#\usepackage{listings}#を使います。これがなければ、エラーを吐かれてしまうので、気を付けましょう。\\

\begin{screen}
\begin{verbatim}

\begin{lstlisting}[オプションを入力する]

内容を書く

\end{lstlisting}

\end{verbatim}
\end{screen}
 というようになります。次に、左に行数が表示された、プログラム全体が四角に囲まれたよく見るパターンのものを紹介します。

\begin{screen}
\begin{verbatim}

\begin{lstlisting}[frame=single, 
basicstyle=\ttfamily, 
numbers=left, 
numbersep=8pt, 
tabsize=2,
extendedchars=true, 
xleftmargin=17pt, 
framexleftmargin=17pt, 
caption=HelloWorld!]
#include<stdio.h>

int main(){
	printf("Hello World!");
}

\end{lstlisting}

\end{verbatim}
\end{screen}
 \verb#となります。今回はオプションが多いのでオプションごとに改行していますが、「,(カンマ)」で区切っていれば改行の必要はありません。また、オプションのあとに一行も開けずコードを書き始めているのは、Listing1:HelloWorld!を見ればわかると思いますが、行数が1のところからプログラムを表示させたかったからです。プログラムの行数は、\begin{lstlisting}の次の行から、書いたプログラムの最後の行までに表示されます。#

\begin{lstlisting}[frame=single, 
basicstyle=\ttfamily, 
numbers=left, 
numbersep=8pt, 
tabsize=2,
extendedchars=true, 
xleftmargin=17pt, 
framexleftmargin=17pt, 
caption=HelloWorld!]
#include<stdio.h>

int main(){
	printf("Hello World!");
}
\end{lstlisting}

\end{document}